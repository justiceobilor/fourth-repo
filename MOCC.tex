
\documentclass[a4paper,12pt]{article}
\begin{document}
\section{Context and Second Generation Models}
In the first generation models by Krugman (1979), the domestic process is exogenous, then the bank reserves bear the adjustment burden to balance of payments. If the authorities pursue inconsistent macroeconomic policies with the fixed exchange rate, there will be a speculative attack that will deplete the foreign reserves and force the government to let the exchange rate to float at a depreciated level. These useful models that properly described the Latin American crises, failed to explain why some countries experienced currency crises despite its governments budget did not have a deficit and the foreign reserves levels were appropriate. 

This was the case of some European countries in the early nineties, for instance Sweden. One year after the Riksbank announced a unilateral peg to the European Currency Unit (ECU), the Danish rejection of Maastricht treaty in 1992 triggered attacks to fixed exchange rates in Europe (specially to Nordic currencies), as well as rose the interest rate. Even though the fundamentals of the Swedish economy were strong enough to resist the attacks for months following the commitment to the long term credibility of the anchor rule, at the end of the year the government decided to let the currency float.

Obstfeld (1994) suggested that first generation models do not take into account the policy options of the government to cope the crises, nor analyzed the costs and trade off of policies and costs of keeping the fixed exchange rate. For these reason the author proposed two models for the so called self-fulfilling crises, in which the government’s responses to market expectations are endogenous.

\subsection{Role of Nominal interest rates}

In a world with two periods, a government can issue domestic currency units and participates in the market for foreign currency units. This government in the period 1, must pay to claimants an amount of domestic currency unit, $D_0^1$, and an amount in period 2 of $D_0^2$. These terms must be understood as the intertemporal endowments of liabilities, hence whether $D_0^1=0$  any the debt is long term, and with $D_0^2=0$ the debt is short term. 

Parallel the government in period 1 is entitled to receive payments of foreign currency, $f_0^1$  ,  and an amount in period 2 of $f_0^2$. Moreover the levels of its consumption in two periods, $g_1$ and $g_2$ are given exogenously. In this model where PPP is assumed, in period 1 the fixed exchange rate (E1) may change in period 2 (E2) . The last assumption is that governments can increase its proceeds by levying taxes in period 2. 

The government has an objective function that takes into account the negative effects of inflation and depreciation, (Ɛ), and tax rate (τ). Firstly the equation is:

$$£=  1/2 τ^2+  θ/2 ε^2            (1) $$

In period 2, the government chooses Ɛ and τ and minimize this loss function subject to a constraint of its position in period 2 when it should meet all the obligations. The revenue to finance the liabilities is obtained from foreign currency assets, taxes, the output (y) and the amount of money hold by resident in period 2 $(M_2-M_1)$. Thus the expression that summarize the constraint is:

$$〖              D〗_1^2+ D_0^2- E_2 (f_1^2+ f_0^2 )+E_2 g_2  = E_2 t y+M_2-M_1               (2) $$

The same equation can be expressed in terms of depreciation rate(ε), and the real value at the of the domestic currency government debt payment promise on date t (d).

$$ε(d_1^2+d_0^2+ky)+ty= d_1^2+d_0^2+g_2-f_1^2-f_0^2     (3)$$

The minimization of (1) subject to (3) allowed to Obstfeld (1994) show that in an optimum, the marginal cost of extra depreciation per domestic currency unit raised equals the marginal cost per domestic currency unit of higher conventional taxes. 

Another minimization of (1) is required, this time subject to two constraints: The first is related to capital mobility and uncovered interest-rate parity that ensures equality of domestic and foreign asset returns :

$$1+i=(E_2/E_1 )  (1+i^*)       (4)$$

The other constraint is the private money demand, whose equation is:

$$M_t=k E_t y      (5)$$

The outcome of the optimizations are the government reaction function curve (the first minimization described), and the interest parity curve (the second minimization). Obstfeld (1994) also calculated the minimizations with a modified loss function of the government when includes the cost of abandoning the fixed exchange rate, associated to political embarrassment and lost of credibility. When this cost surpassed by the loss, the government will be force to devalue.

The interception of the curves displays that having multiple equilibriums is likely. Regarding the latter, it can be considered a scenario where the market do not expect a devaluation and actually there is no devaluation so it is possible to keep the fix exchange rate. If suddenly the market participants think that the government is trying to surprise the market with a depreciation in order to rise economic growth (by reducing the interest rates), the expectation leads to a real devaluation since augments the cost of keeping the peg (Eijffinger and Goderis, 2007).  In this sense, this result is different to the obtained in the first generation model, once that the lasts only may reach one equilibrium . 

\subsection{Role of aggregate demand shocks} 
The second model of Obstfeld (1994) shows that a regime of fixed but adjustable parities can engender multiple equilibria and in some of them the economy will be worse than under fixed exchange rates, owing to expectation of growth of wage that negatively affect the competitiveness and force devaluations. 

In this model of aggregate demand, where PPP is assumed, the domestic output $y_t$is:

$$y_t= ∝(e_t-w_t )- u_t       (1)$$

Where $e_t$ is the home-currency price of the foreign exchange $w_t$ is money wage and u is a mean zero. In this equation can be seen that the government only is able to respond to demand shocks is by changing the exchange rate in date t, since the wages were already fixed. Now the government tries to minimize a loss function (as was seen in the first model), this time incorporate level of exchange rate and output, the equation is the following:

$$£_t=∑_(s=t)^∝▒β^(s-t)  [θ(e_s-e_(s-t) )+〖(y_s-y^*)〗^2 ]       (2)$$

This loss function can be combined with (1), and the government must choose the currency exchange rate Government chooses currency's exchange rate as it was mentioned to minimize l (over e_t) given the nominal wages agreed in period t-1


$$l_t=  θ/2 〖(e_t-e_(1-l))〗^2+1/2 [∝(e_t-w_t )- u_t-y*]^2     (3)$$

After the minimization, Obstfeld (1994) showed that the goverment's desire to offset negative shocks can trigger a devaluation that would not have occurred under different private expectations. 

The defensive measures to defend the fixed exchange rate, such as, raising interest rates to cope the expectations of devaluation, borrowing foreign reserves, reduce the government’s borrowing requirements or Impose exchange controls, although effective, are costly, especially when the economies have trade off related to economic downturn or high unemployment rates.

The second generation models allow to see that the dynamic preferences and constraints of the governments are endogenous through the market expectations, so a country with a peg or semi pegged system may face a currency crisis even if it has strong fundamentals, once that it might become too costly to try to maintain the fixed exchange rate. 

\subsection {Criticism to first and second generation models}
The shortcomings of the first and second generation models were seen in the context of the Asian crisis, then the recent literature have focused on the improvement of the models. For instance Kaminsky & Reinhart (1999) pointed that despite some Asian countries had a stable inflation, an economic boom and fiscal surpluses, nevertheless these countries experienced a severe currency crisis owing to factors such as weak regulation of the banking system. These variables, which are omitted in the first and second generation models, may have undesirable consequences over the balance sheets of highly leverage companies or the banking system. Burnside, Eichenbaum and Rebelo (2001a) stated that that a currency crisis could begin even before the government actually starts to print money (as the first generation model predicts) since other aspects must be taken into account and Krugman (1999) highlighted the need to incorporate variables related to financial fragility.   

\subsection{Conclusions}
The First generation models which was given birth to by the work of Krugman attempted to provide explanations regarding the Latin Debt Crisis. The key characteristic in the first generation model is that a small open economy under a fixed exchange rate regime will switch to a flexible or floating exchange as a result of a speculative attack on the domestic currency. As discussed above its assumptions could not hold ground in explaining future crisis, however the models assisted in future research work.

The second generation models developed by Obstfeld (1994), provided explanations to the currency crisis of some European countries and improve the first generation models since they incorporated the notion of tradeoff between the maintenance of a fixed Exchange rate and pursuing expansionary monetary policies. Moreover the result showed that it is possible to have multiple equilibrium given the dynamic preferences of the governments. 

The first and second generation models were useful theoretical frameworks to understand currency crises, although some progress have been made since the Asian crisis of 1997. The third generation models include the soundness of the financial system, as well as the existence of appropriate regulation (among other aspects), in order to explain in a better way this kind of economic crisis, which are still likely even when many countries have opted for a float Exchange rate regime. 


 
\subsection{References}

Eijffinger, S. C. W., & Goderis, B. V. G. (2007). Currency crises, monetary policy and corporate balance sheet vulnerabilities. German Economic Review, 8(3), 309-343.

Eijffinger, S. C., & Karataş, B. (2012). Currency crises and monetary policy: A study on advanced and emerging economies. Journal of International Money and Finance, 31(5), 948-974.

Kaminsky, G. L., & Reinhart, C. M. (1999). The twin crises: the causes of banking and balance-of-payments problems. American economic review, 473-500.

Krugman, P (1979). "A model of balance-of-payments crises." Journal of money, credit and banking 11.3 311-325

Krugman, P. (1999). Balance Sheets, the Transfer Problem, and Financial Crises. International Tax and Public Finance, 6, 459-472.

Sarno, L., & Taylor, M. P. (2002). The economics of exchange rates. Cambridge University Press.

The Economist (2016). Hope the naira falls 

Flood, R. P., Garber, P. M., & Kramer, C. (1996). Collapsing exchange rate regimes: Another linear example. Journal of International Economics, 41(3), 223-234.



\end{document}